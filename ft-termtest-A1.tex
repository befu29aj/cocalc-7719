% ft-termtest-A1.tex

\documentclass[11pt]{article}
\usepackage{alltt}
\usepackage{enumerate}
\usepackage{syllogism} 
\usepackage{october}
\usepackage[table]{xcolor}
\pagestyle{empty}

\newcounter{aufg}
\setcounter{aufg}{0}
\newcommand{\aufgabe}[1]{\refstepcounter{aufg}\textbf{(\arabic{aufg})} [#1 points]}

\begin{document}

\textbf{Term Test A version 1}

\aufgabe{5} Solution X is a 27 percent salt solution and Solution Y is a 20 percent salt solution. How much of each is needed to make 42 gallons of a 25 percent salt solution?

\aufgabe{5} Solve the equation.

\begin{equation}
\label{eq:oopeecha}
(2x-3)^{2}+(2x-4)^{2}=4(x-1)^{2}\notag
\end{equation}

\aufgabe{5} Solve the equation.

\begin{equation}
\label{eq:ulugheec}
\frac{3+x}{2}-\frac{2x-7}{3}=3\notag
\end{equation}

\aufgabe{5} Two train stations $A$ and $B$ are 310 kilometres apart. The first train leaves $A$ at 6:30am going towards $B$. The second train leaves $B$ at 7:20am going towards $A$. The velocity of the first train is 10 kilometres per hour less than the velocity of the second train. At 8:50am the trains are still 65 kilometres apart. Calculate the speed of the two trains and when they will meet. Use $v\cdot{}t=s$ (velocity times time equals distance).

\aufgabe{5} You have 20 gallons of a 45 percent antifreeze solution. How many gallons of a 57 percent antifreeze solution needs to be added to make a 51 percent antifreeze solution?

\aufgabe{5} The formula to work out the total resistance $R_{T}$ given two resistors $R_{1}$ and $R_{2}$ in parallel as in the diagram is
\begin{equation}
\label{eq:tiexueri}
\frac{1}{R_{T}}=\frac{1}{R_{1}}+\frac{1}{R_{2}}\notag
\end{equation}

\begin{figure}[ht]
\includegraphics[scale=.7]{./diagrams/resist.png}
\end{figure}

The total resistance has been measured at 3 ohms, and one of the resistors is known to be 8 ohms more than the other. Ohm is the unit for resistance, and only a positive number of ohms makes sense. Calculate $R_{1}$.

\end{document}
