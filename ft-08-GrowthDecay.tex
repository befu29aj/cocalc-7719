% ft-08-GrowthDecay.tex

\documentclass[xcolor=dvipsnames]{beamer}
\usepackage{teachbeamer}

\title{Growth and Decay}
\subtitle{{\CourseNumber}, BCIT}

\author{\CourseName}

\date{October 18, 2018}

\begin{document}

\begin{frame}
  \titlepage
\end{frame}

\begin{frame}
  \frametitle{Radiocarbon Dating}
Radiocarbon dating is a method archeologists use to determine the age
of ancient objects. The carbon dioxide in the atmosphere always
contains a fixed fraction of radioactive carbon, carbon-14 ($^{14}$C),
with a half-life of about 5730 years. Plants absorb carbon dioxide
from the atmosphere, which then makes its way to animals through the
food chain. Thus, all living creatures contain the same fixed
proportions of $^{14}$C to nonradioactive $^{12}$C as the atmosphere.

\medskip

After an organism dies, it stops assimilating $^{14}$C, and the amount
of $^{14}$C in it begins to decay exponentially. We can then determine
the time elapsed since the death of the organism by measuring the
amount of $^{14}$C left in it.
\end{frame}

\begin{frame}
  \frametitle{Radiocarbon Dating Example}
If a donkey bone contains 73\% as much $^{14}$C as a living donkey,
when did it die?

\medskip

Look at the following table to notice the pattern,
\begin{equation}
  \label{eq:sayeeziu}
  \begin{array}{rcccl}
        100\% & \ldots & 2^{-0} & \ldots & 5730\cdot{}0 \\
        50\% & \ldots & 2^{-1} & \ldots & 5730\cdot{}1 \\
        25\% & \ldots & 2^{-2} & \ldots & 5730\cdot{}2
  \end{array}
\end{equation}
Therefore ($t$ being the number of years ago that the donkey died),
\begin{equation}
  \label{eq:anuvuama}
  t=\left(-\log_{2}0.73\right)\cdot{}5730=-\frac{\ln{}0.73}{\ln{}2}\cdot{}5730\approx{}2600
\end{equation}
\end{frame}

\begin{frame}
  \frametitle{Compound Interest 1}
Now use a similar strategy calculating compound interest. Consider a
$\$5,000$ loan at 6\% per annum (p.a.). 
\begin{equation}
  \label{eq:booyiezi}
  \begin{array}{rcccl}
        \mbox{after }0\mbox{ year(s)} & \ldots & \$5,000\cdot{}1.06^{0} & \ldots & \$5,000.00 \\
        \mbox{after }1\mbox{ year(s)} & \ldots & \$5,000\cdot{}1.06^{1} & \ldots & \$5,300.00 \\
        \mbox{after }2\mbox{ year(s)} & \ldots & \$5,000\cdot{}1.06^{2} & \ldots & \$5,618.00 \\
        \mbox{after }3\mbox{ year(s)} & \ldots & \$5,000\cdot{}1.06^{3} & \ldots & \$5,955.08 \\
  \end{array}
\end{equation}
\end{frame}

\begin{frame}
  \frametitle{Compound Interest 2}
Derive the \alert{compound interest formula},
\begin{equation}
  \label{eq:paeniawa}
  A=P\left(1+\frac{r}{m}\right)^{mt}
\end{equation}
\begin{tabular}{rcl}
  $A$&\ldots&accumulated amount at the end of $t$ years \\
  $P$&\ldots&principal \\
  $r$&\ldots&interest rate p.a. \\
  $m$&\ldots&number of conversion periods per year \\
  $t$&\ldots&term (number of years)
\end{tabular}
\end{frame}

\begin{frame}
  \frametitle{Compound Interest 3}
{\ubung} Find the accumulated amount after 3 years if \$1,000 is invested at
8\% per year compounded (a) annually, (b) semi-annually, (c)
quarterly, (d) monthly, and (e) daily. 

\medskip

Look at the pattern and notice how it leads to the \alert{continuous
  compound interest formula},
\begin{equation}
  \label{eq:johsheth}
  A=Pe^{rt}
\end{equation}
\end{frame}

\begin{frame}
  \frametitle{Compound Interest Exercises}
{\ubung} Find the amount that results from each investment; that is, find the
\alert{future value}. All percentages are per annum.
\begin{enumerate}
\item \$100 invested at 4\% compounded quarterly after a period of 2
  years.
\item \$500 invested at 8\% compounded quarterly after a period of 2.5
  years.
\item \$3,000 invested at 5\% compounded annually after a period of 20
  years.
\item \$1,000 invested at 10\% compounded continuously after a period of 2$\frac{1}{4}$
  years.
\end{enumerate}
\end{frame}

\begin{frame}
  \frametitle{Compound Interest Exercises}
{\ubung} Find the principal needed now to get each amount; that is, find the
\alert{present value}. All percentages are per annum.
\begin{enumerate}
\item To get \$300 after four years at 3\% compounded daily.
\item To get \$75,000 after three years at 8\% compounded quarterly.
\item To get \$400 after one year at 10\% compounded continuously.
\item To get \$1,000,000 after two years at 6\% compounded semi-annually.
\end{enumerate}
\end{frame}

\begin{frame}
  \frametitle{Compound Interest Exercises}
{\ubung} Answer the following questions.
  \begin{enumerate}
  \item What rate of interest compounded annually is required to double an investment in three years?
  \item What rate of inflation doubles prices every 14 years?
    (Inflation is like interest compounded annually.)
  \item John will require \$3,000 in 6 months to pay off a loan
    that has no prepayment privileges. If he has the \$3,000 now, how
    much of it should he save in an account paying 3\% compounded
    monthly so that in six months he will have exactly \$3,000?
  \item A business purchased for \$650,000 in 1994 is sold in 1997
    for \$850,000. What is the annual rate of return for this investment?
  \end{enumerate}
\end{frame}

\begin{frame}
  \frametitle{Uninhibited Growth and Decay Formula}
Many natural phenomena have been found to follow the law that an
amount $A$ varies with time $t$ according to
\begin{equation}
  \label{eq:peifaiva}
  A(t)=A_{0}e^{kt}
\end{equation}
where $A_{0}=A(0)$ is the original amount at $t=0$ and $k\neq{}0$ is a
constant. If $k>0$, there is growth. If $k<0$, there is decay.
\end{frame}

\begin{frame}
  \frametitle{Uninhibited Growth and Decay Exercise}
{\ubung} A colony of bacteria grows according to the law of uninhibited growth
according to the function
\begin{equation}
  \label{eq:chiowezo}
  N(t)=100e^{0.045t}
\end{equation}
where $N$ is measured in grams and $t$ is measured in days.
  \begin{enumerate}
  \item Determine the initial amount of bacteria.
  \item What is the growth rate of the bacteria?
  \item What is the population after five days?
  \item How long will it take for the population to reach 140
    grams?
  \item What is the doubling time for the population?
  \end{enumerate}
\end{frame}

\begin{frame}
  \frametitle{Newton's Law of Cooling}
The temperature $u$ of a heated object at a given time $t$ can be
modeled by the following function,
\begin{equation}
  \label{eq:iemahbec}
  u(t)=T+(u_{0}-T)e^{kt}
\end{equation}
where $k$ is a negative constant, $T$ is the constant temperature of
the surrounding medium, and $u_{0}$ is the initial temperature of the
heated object.
\end{frame}

\begin{frame}
  \frametitle{Newton's Law of Cooling Exercise}
{\ubung} An object is heated to 100$^{\circ}C$ (degrees Celsius) and is then
allowed to cool in a room whose air temperature is 30$^{\circ}C$.
\begin{enumerate}
\item If the temperature of the object is $80^{\circ}C$ after five
  minutes, when will its temperature be $50^{\circ}C$?
\item Determine the elapsed time before the temperature of the
  object is $35^{\circ}C$.
\item What do you notice about $u(t)$, the temperature, as $t$,
  time, passes?
\end{enumerate}
\end{frame}

\begin{frame}
  \frametitle{Newton's Law of Cooling Exercise}
{\ubung} A frozen steak has a temperature of $28^{\circ}F$. It is placed in a
room with a constant temperature of $70^{\circ}F$. After 10 minutes,
the temperature of the steak has risen to $35^{\circ}F$. 
\begin{enumerate}
\item<1-> What will the temperature of the steak be after 30 minutes?
\item<2-> How long will it take the steak to thaw to a temperature of $45^{\circ}F$?
\end{enumerate}
\end{frame}

\begin{frame}
  \frametitle{Newton's Law of Cooling Exercise}
{\ubung} The hotel Bora-Bora is having a pig roast. At noon, the chef put the
pig in a large earthen oven. The pig's original temperature was
$75^{\circ}F$. At 2:00\textsc{pm} the chef checked the pig's
temperature and was upset because it had reached only $100^{\circ}F$.
\begin{enumerate}
\item If the oven's temperature remains a constant $325^{\circ}F$, at
  what time may the hotel serve its guests, assuming that pork is done
  when it reaches $175^{\circ}F$?
\end{enumerate}
\end{frame}

\begin{frame}
  \frametitle{Exercises}
{\ubung}  A small lake is stocked with a certain species of fish. The fish
  population is modeled by the function
  \begin{equation}
    \label{eq:peongeex}
    P=\frac{10}{1+4e^{-0.8t}}
  \end{equation}
where $P$ is the number of fish in thousands and $t$ is measured in
years since the lake was stocked.
\begin{enumerate}
\item Find the fish population after 3 years.
\item After how many years will the fish population reach 5000 fish?
\end{enumerate}
\end{frame}

\begin{frame}
  \frametitle{Exercises}
  {\ubung} A culture starts with 8600 bacteria. After one hour the
  count is $10,000$.
\begin{enumerate}
\item Find a function that models the number of bacteria $n(t)$ after $t$ hours.
\item Find the number of bacteria after 2 hours.
\item After how many hours will the number of bacteria double?
\end{enumerate}
\end{frame}

\begin{frame}
  \frametitle{Exercises}
  {\ubung} Atmospheric pressure decreases exponentially as you go higher
above sea level. It decreases about 12\% for every 1000 metres. The
pressure at sea level is 1013 hectopascal (hPa). The formula is
\begin{equation}
  \label{eq:oopohtee}
  y(t)=y(0)\cdot{}e^{ks}
\end{equation}
where $s$ is the distance above sea level in metres and $k$ is a
positive constant. What is your prediction for the air pressure on
Mount Everest (8848 metres above sea level)?
\end{frame}

\begin{frame}
  \frametitle{Exercises}
  {\ubung} The number of people living in a country is increasing each year
exponentially. The number of people 5 years ago was 4 million. The
number of people in five years is projected to be 6.25 million. What
is the present population of the country?
\end{frame}

\begin{frame}
  \frametitle{Exercises}
  {\ubung} ``Loudness'' is measured in decibels. The formula for the loudness
of a sound is given by 
\begin{equation}
\label{eq:thaixazu}
L=10\log\frac{I}{I_{0}}
\end{equation}
where $I_{0}$ is the intensity of ``threshold sound,'' or sound that
can barely be perceived. The logarithm here is to base $10$, i.e.\ the
common logarithm. Other sounds are defined in terms of how many times
more intense they are than threshold sound. For instance, a cat's purr
is about 316 times as intense as threshold sound, for a decibel rating
of:
\begin{equation}
  \label{eq:ahgheeha}
  10\log\frac{I}{I_{0}}=10\log\frac{316\cdot{}I_{0}}{I_{0}}=10\log{}316\approx{}25\mbox{ decibels}
\end{equation}
\end{frame}

\begin{frame}
  \frametitle{Exercises}
  \begin{enumerate}
  \item An airplane jet take-off heard from 100 metres away measures 100
decibels. How much more intense than a normal conversation (50
decibels) is the sound of an airplane jet take-off? (Do not use
exponents in your answer.)
\item Prolonged exposure to sounds above 85 decibels can cause hearing
damage or loss. What is the approximate loudness of a gunshot from a
.22 rimfire rifle with an intensity of about
$I=(2.5\cdot{}10^{13})I_{0}$? Should you follow the rules and wear ear
protection when visiting the rifle range?
  \end{enumerate}
\end{frame}

\begin{frame}
  \frametitle{End of Lesson}
Next Lesson: Basic Rules of Differentiation
\end{frame}

\end{document}

