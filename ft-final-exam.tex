% ft-final-exam.tex

\documentclass[11pt]{article}
% \usepackage{enumerate}
% \usepackage{syllogism} 
\usepackage{october}
% \usepackage[table]{xcolor}
\pagestyle{empty}
\newcounter{frage}
\setcounter{frage}{0}
\newcommand{\frage}[0]{\refstepcounter{frage}\arabic{frage}}

\begin{document}

\textbf{Final Exam}

Show all of your work. Correct answers without showing how to get them
does not earn you points.

({\frage}) Solve the logarithmic equation. Do not use a calculator for your
calculation. It's OK to use a calculator to check your answer.

\begin{equation}
  \label{eq:vuomebic}
  \log_{2}x-\log_{2}\frac{1}{4}=3
\end{equation}

({\frage}) Solve the exponential equation. Do not use a calculator for your
calculation. It's OK to use a calculator to check your answer.

\begin{equation}
  \label{eq:vaiphook}
  3^{x-1}=e^{3x}  
\end{equation}

({\frage}) Use Newton's method to find an $x$-intercept for $f(x)=e^{x}-10x$.
Show all of your work and all intermediate steps!

({\frage}) Find the area under the curve $y=(x^{2}-1)x$ between $x=-1$ and
$x=0$.

({\frage}) Differentiate

\begin{equation}
  \label{eq:ohzaingu}
  f(x)=(x+4)\sqrt{x-1}  
\end{equation}

({\frage}) What are the critical points of the function

\begin{equation}
  \label{eq:aoshiehi}
  g(s)=(4s^{2}-15s-18)\frac{1}{6}s
\end{equation}

Specify whether these critical points are maxima or minima.

({\frage}) Forensic scientists determine the temperature $T$ (in $^{\circ}C$)
of a body $t$ hours after death from the equation 
\begin{equation}
\label{eq:aehoosut}
T=T_{0}+(37-T_{0})0.87^{t}
\end{equation}
where $T_{0}$ is the air temperature. If a body is discovered at
midnight with a body temperature of 27$^{\circ}C$ in a room with air
temperature 22$^{\circ}C$, at what time did death occur (use the
hour-minute format for your answer, for \mbox{example} 7:42pm)?

\end{document}

