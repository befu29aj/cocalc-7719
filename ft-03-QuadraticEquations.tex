% ft-03-QuadraticEquations.tex

\documentclass[xcolor=dvipsnames]{beamer}
\usepackage{teachbeamer}

\title{Quadratic Equations}
\subtitle{{\CourseNumber}, BCIT}

\author{\CourseName}

\date{September 17, 2018}

\begin{document}

\begin{frame}
  \titlepage
\end{frame}

\begin{frame}
  \frametitle{What Is a Quadratic Equation}
A quadratic equation is an equation that can be reduced to the form
\begin{equation}
  \label{eq:paezohmu}
  ax^{2}+bx+c=0\mbox{ with }a,b,c\in\mathbb{R}\mbox{ and }a\neq{}0
\end{equation}
$a$, $b$ and $c$ are called the \emph{coefficients} of the quadratic
equation.
\end{frame}

\begin{frame}
  \frametitle{Completing the Square I}
In order to solve quadratic equations, we use the following identity,
\begin{equation}
  \label{eq:uibaifoa}
  (r+s)^{2}=r^{2}+2rs+s^{2}
\end{equation}
\end{frame}

\begin{frame}
  \frametitle{Completing the Square II}
If $a=1$, then
\begin{equation}
  \label{eq:aiphohpi}
  x^{2}+bx+c=\left(x+\frac{1}{2}b\right)^{2}+\left(-\frac{1}{4}b^{2}+c\right)
\end{equation}
Therefore, the solution set for the equation $x^{2}+bx+c=0$ is the
same as the solution set for the equation
\begin{equation}
  \label{eq:diesaisi}
  \left(x+\frac{1}{2}b\right)^{2}=\left(\frac{1}{4}b^{2}-c\right)
\end{equation}
\end{frame}

\begin{frame}
  \frametitle{Interlude}
Here is a simple quadratic equation. Let $a=1,b=0,c=-d$, so
\begin{equation}
  \label{eq:nuphoomu}
  x^{2}=d
\end{equation}
If $d>0$, then there are two solutions for this equation!
\begin{equation}
  \label{eq:gephoopo}
  x_{1}=\sqrt{d}\mbox{ and }x_{2}=-\sqrt{d}
\end{equation}
The solution set is $S=\left\{\sqrt{d},-\sqrt{d}\right\}$. If $d=0$,
then $S=\{0\}$, and if $d<0$, then $S=\{\}$.
\end{frame}

\begin{frame}
  \frametitle{Completing the Square III}
Let's look at an example.
\begin{equation}
  \label{eq:chohceih}
  x^{2}+6x+5=0
\end{equation}
Completing the square, we get
\begin{equation}
  \label{eq:raibohph}
  (x+3)^{2}=4
\end{equation}
Remember the two solutions for equation (\ref{eq:nuphoomu})?
There are two solutions here as well,
\begin{equation}
  \label{eq:efehisuu}
  x+3=2\mbox{ and }x+3=-2
\end{equation}
so the solution set is $S=\{-5,-1\}$.
\end{frame}

\begin{frame}
  \frametitle{The Quadratic Formula}
Completing the square for the general quadratic equation
$ax^{2}+bx+c=0$ gives us the \emph{quadratic formula},
\begin{equation}
  \label{eq:iedohkei}
\begin{array}{lcl}
x_{1}=\frac{-b+\sqrt{b^{2}-4ac}}{2a} & \hspace{.5in} & x_{2}=\frac{-b-\sqrt{b^{2}-4ac}}{2a}
\end{array}
\end{equation}
$\{x_{1},x_{2}\}$ is the solution set for the quadratic equation. The
expression under the root sign, $b^{2}-4ac$, is called the
\emph{discriminant} of the quadratic equation. Explain why the
quadratic equation has no solutions in the real numbers if the
discriminant $D<0$, one solution if $D=0$, and two solutions if $D>0$.
\end{frame}

\begin{frame}
  \frametitle{Exercises}
{\ubung} Solve the following equations,
\begin{equation}
  \label{eq:humourah}
  \begin{array}{rclcrcl}
    x^{2}+2x-2&=&0&\hspace{.5in}&3x^{2}-6x-1&=&0 \\
\hspace{.5in} && \\
    x^{2}+12x-27&=&0&\hspace{.5in}&8x^{2}-6x-9&=&0 \\
\hspace{.5in} && \\
    3x^{2}+6x-5&=&0&\hspace{.5in}&2y^{2}-y-\frac{1}{2}&=&0 \\
  \end{array}
\end{equation}
\end{frame}

\begin{frame}
  \frametitle{Exercises}
{\ubung} Solve the following equations,
\begin{equation}
  \label{eq:koaseich}
  \begin{array}{rcl}
    x^{2}-6x+1&=&0 \\
\hspace{.5in} && \\
    \frac{x}{2x+7}-\frac{x+1}{x+3}&=&1 \\
\hspace{.5in} && \\
    2x+\sqrt{x+1}&=&8 
  \end{array}
\end{equation}
\end{frame}

\begin{frame}
  \frametitle{Exercises}
  \addtocounter{exercise}{-1}
{\ubung} Solve the following equations,
\begin{equation}
  \label{eq:neeshouw}
  \begin{array}{rcll}
    x^{2}-6x+1&=&0&\alert{S=\left\{3+2\sqrt{2},3-2\sqrt{2}\right\}} \\
\hspace{.5in} && \\
    \frac{x}{2x+7}-\frac{x+1}{x+3}&=&1&\alert{S=\left\{-4,-\frac{7}{3}\right\},D=361-336=25} \\
\hspace{.5in} && \\
    2x+\sqrt{x+1}&=&8&\alert{S=\left\{3\right\},D=1089-1008=81}
  \end{array}
\end{equation}
\end{frame}

\begin{frame}
  \frametitle{Word Problems for Quadratic Equations}
{\ubung} A motorboat makes a round trip on a river 56 miles upstream and 56
miles downstream, maintaining the constant speed 15 miles per hour
relative to the water. The entire trip up and back takes 7.5 hours.
What is the speed of the current?
\end{frame}

\begin{frame}
  \frametitle{Word Problems for Quadratic Equations}
{\ubung} The product of two consecutive negative integers is 1122. What are
the numbers?
\end{frame}

\begin{frame}
  \frametitle{Word Problems for Quadratic Equations}
{\ubung} A garden measuring 12 meters by 16 meters is to have a pedestrian
pathway installed all around it, increasing the total area to 285
square meters. What will be the width of the pathway? See diagram.
\end{frame}

\begin{frame}
  \frametitle{Word Problems for Quadratic Equations}
{\ubung} A fast train runs 8 mi/h faster than a slow train and takes 3 hours
less to travel 288 miles. Find the rates of the trains.
\end{frame}

\begin{frame}
  \frametitle{Word Problems for Quadratic Equations}
{\ubung} The length of a rectangle court exceeds its width by 2 metres. If the
length and the width were each increased by 3 metres, the area of the
court would be 80 m$^{2}$. Find the dimensions of the court.
\end{frame}

\begin{frame}
  \frametitle{End of Lesson}
Next Lesson: Percent and Mixtures
\end{frame}

\end{document}

