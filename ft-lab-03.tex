% ft-lab-03.tex

\documentclass[11pt]{article}
\usepackage{enumerate}
\usepackage{syllogism} 
\usepackage{october}
\usepackage[table]{xcolor}

\newcounter{aufg}
\setcounter{aufg}{0}
\newcommand{\aufgabe}[0]{\refstepcounter{aufg}\textbf{(\arabic{aufg})}}

\begin{document}

\textbf{Exponents and Logarithms}

{\aufgabe} Simplify the following expression,
\begin{equation}
  \label{eq:zaiquahu}
  \left(64^{\frac{4}{3}}\right)^{-\frac{1}{2}}\notag
\end{equation}

{\aufgabe} Simplify the following expression,
\begin{equation}
  \label{eq:uobeigha}
  (16\cdot{}81)^{-\frac{1}{4}}\notag
\end{equation}

{\aufgabe} Simplify the following expression,
\begin{equation}
  \label{eq:neeshaej}
  \left(\frac{3^{\frac{1}{2}}}{2^{\frac{1}{3}}}\right)^{4}\notag
\end{equation}

{\aufgabe} Simplify the following expression,
\begin{equation}
  \label{eq:taejahbo}
\sqrt[3]{108}-\sqrt[3]{32}\notag
\end{equation}

{\aufgabe} Simplify the following expression,
\begin{equation}
  \label{eq:woongais}
(3ab^{2}c)\left(\frac{2a^{2}b}{c^{3}}\right)^{-2}\notag
\end{equation}

  {\aufgabe} Simplify the following expression,
\begin{equation}
  \label{eq:eajaekei}
  \left(\frac{x^{-3}}{y^{-2}}\right)^{2}\left(\frac{y}{x}\right)^{4}\notag
\end{equation}

  {\aufgabe} Simplify the following expression,
\begin{equation}
  \label{eq:ohchoewi}
  \sqrt[3]{x^{-2}}\cdot\sqrt{4x^{5}}\notag
\end{equation}

  {\aufgabe} Evaluate the following expression,
\begin{equation}
  \label{eq:ohzeiphi}
  \left(\frac{7^{-5}\cdot{}7^{2}}{7^{-2}}\right)^{-1}\notag
\end{equation}

  {\aufgabe} Evaluate the following expression,
\begin{equation}
  \label{eq:puareipo}
  \sqrt[3]{\frac{-8}{27}}\notag
\end{equation}

{\aufgabe} Show that 
\begin{equation}
\label{eq:nuocaeph}
-\ln{}\left(x-\sqrt{x^{2}-1}\right)=\ln{}\left(x+\sqrt{x^{2}-1}\right)\notag
\end{equation}

{\aufgabe} Use the Change of Base Formula and the calculator to evaluate
$\mbox{log}_{7}24$ and $\mbox{log}_{3}59049$.

{\aufgabe} Rewrite the expression as a single logarithm,
\begin{equation}
  \label{eq:zahshaum}
\ln(a+b)+\ln(a-b)-2\ln{}c\notag
\end{equation}

{\aufgabe} Analyze the expression so there is no longer a logarithm of a
product, quotient, root, or power: 
\begin{equation}
  \label{eq:ooreyonu}
  \log\left(\frac{a^{2}}{b^{4}\sqrt{c}}\right)\notag
\end{equation}

  {\aufgabe} Solve the equation.
  \begin{equation}
    \label{eq:rohkiine}
    4^{1-2x}=2\notag
  \end{equation}

  {\aufgabe} Solve the equation.
  \begin{equation}
    \label{eq:iaphaeya}
    3^{x^{2}+x}=\sqrt{3}\notag
  \end{equation}

  {\aufgabe} Solve the equation.
  \begin{equation}
    \label{eq:iebaiviu}
    4^{x-x^{2}}=\frac{1}{2}\notag
  \end{equation}

  {\aufgabe} Solve the equation.
  \begin{equation}
    \label{eq:maareiju}
    \log_{x}64=-3\notag
  \end{equation}

  {\aufgabe} Solve the equation.
  \begin{equation}
    \label{eq:dairithe}
    5^{x}=3^{x+2}\notag
  \end{equation}

  {\aufgabe} Solve the equation.
  \begin{equation}
    \label{eq:xiefepib}
    \log_{3}\sqrt{x-2}=2\notag
  \end{equation}

  {\aufgabe} Solve the equation.
  \begin{equation}
    \label{eq:eegaifah}
    2^{x+1}\cdot{}8^{-x}=4\notag
  \end{equation}

  {\aufgabe} Solve the equation.
  \begin{equation}
    \label{eq:aeshaite}
    \log_{6}(x+3)+\log_{6}(x+4)=1\notag
  \end{equation}

  {\aufgabe} Solve the equation.
  \begin{equation}
    \label{eq:niephait}
    e^{1-x}=5\notag
  \end{equation}
  {\aufgabe} Solve the equation.
  \begin{equation}
    \label{eq:eevaicei}
    2^{3x}=3^{2x+1}\notag
  \end{equation}

  {\aufgabe} Solve the equation.
\begin{equation}
  \label{eq:haesiiro}
  e^{2x}-e^{x}-6=0\notag
\end{equation}


\end{document}

