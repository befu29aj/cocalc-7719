% ft-04-PercentMixtures.tex

\documentclass[xcolor=dvipsnames]{beamer}
\usepackage{teachbeamer}

\title{Percent and Mixtures}
\subtitle{{\CourseNumber}, BCIT}

\author{\CourseName}

\date{September 25, 2018}

\begin{document}

\begin{frame}
  \titlepage
\end{frame}

\begin{frame}
  \frametitle{Percent}
The definition of percent is similar to the definition of the word
``quarter.'' When I say, ``three quarters,'' I mean $\frac{3}{4}$.
When I say ``sixty-two percent,'' I mean $\frac{62}{100}$. Percent is
not a unit---it simply means that the number in question is divided by
one hundred. Have a look at the following table.
\end{frame}

\begin{frame}
  \frametitle{Percent}
\begin{tabular}{|l|r|}\hline
  $0.12$ & 12\% \\ \hline
  three quarters & 75\% \\ \hline
  $0.75$ & 75\% \\ \hline
  one half & 50\% \\ \hline
  $0.5$ & 50\% \\ \hline
  one and a half & 150\% \\ \hline
  $1.3$ & 130\% \\ \hline
\end{tabular}
\end{frame}

\begin{frame}
  \frametitle{Mixtures}
  You need a 15\% acid solution for a certain test, but your supplier
  only ships a 10\% solution and a 30\% solution. Rather than pay the
  hefty surcharge to have the supplier make a 15\% solution, you
  decide to mix 10\% solution with 30\% solution, to make your own
  15\% solution. You need 10 litres of the 15\% acid solution. How
  many litres of 10\% solution and 30\% solution should you use?
\end{frame}

\begin{frame}
  \frametitle{Mixtures}
  Let $x$ stand for the number of litres of 10\% solution, and let $y$
  stand for the number of litres of 30\% solution. For mixture
  problems, it is often helpful to create a table:

  \bigskip

  \begin{tabular}{|l|c|c|c|}\hline
    & litres solution & percent acid & total litres acid \\ \hline
    10\% solution & $x$ & 0.10 & $0.10x$ \\ \hline
    30\% solution & $y$ & 0.30 & $0.30y$ \\ \hline
    mixture & $x+y=10$ & 0.15 & $0.15\cdot{}10=1.5$ \\ \hline
  \end{tabular}
\end{frame}

\begin{frame}
  \frametitle{Mixtures}
  Since $x+y=10$, then $x=10-y$. Using this, we can substitute for $x$
  in our grid, and eliminate one of the variables:

  \bigskip

  \begin{tabular}{|l|c|c|c|}\hline
    & litres solution & percent acid & total litres acid \\ \hline
    10\% solution & $10-y$ & 0.10 & $0.10\cdot(10-y)$ \\ \hline
    30\% solution & $y$ & 0.30 & $0.30y$ \\ \hline
    mixture & $x+y=10$ & 0.15 & $0.15\cdot{}10=1.5$ \\ \hline
  \end{tabular}
\end{frame}

\begin{frame}
  \frametitle{Mixtures}
  When the problem is set up like this, you can usually use the last
  column to write your equation. The litres of acid from the 10\%
  solution, plus the litres of acid in the 30\% solution, add up to
  the litres of acid in the 15\% solution. Then:
  \begin{equation}
    \label{eq:voocoshi}
    \begin{array}{rcl}
      0.10\cdot(10-y)+0.30y&=&1.5 \\
      1-0.10y+0.30y&=&1.5 \\
      1+0.20y&=&1.5 \\
      0.20y&=&0.5 \\
      y&=&2.5
    \end{array}
  \end{equation}
  Then we need 2.5 litres of the 30\% solution, and
  $x=10-y=10-2.5=7.5$ litres of the 10\% solution. If you think about
  it, this makes sense. Fifteen percent is closer to 10\% than to
  30\%, so we ought to need more 10\% solution in our mix.
\end{frame}

\begin{frame}
  \frametitle{Mixture and Percentage Word Problems}
  {\ubung} How many litres of a 70\% alcohol solution must be added to
  50 litres of a 40\% alcohol solution to produce a 50\% alcohol
  solution?
\end{frame}

\begin{frame}
  \frametitle{Mixture and Percentage Word Problems}
  {\ubung} How many ounces of pure water must be added to 50 ounces of
  a 15\% saline solution to make a saline solution that is 10\% salt?
\end{frame}

\begin{frame}
  \frametitle{Mixture and Percentage Word Problems}
  {\ubung} Find the selling price per pound of a coffee mixture made
  from 8 pounds of coffee that sells for \$9.20 per pound and 12
  pounds of coffee that costs \$5.50 per pound.
\end{frame}

\begin{frame}
  \frametitle{Mixture and Percentage Word Problems}
  {\ubung} How many pounds of lima beans that cost \$0.90 per pound
  must be mixed with 16 pounds of corn that costs \$0.50 per pound to
  make a mixture of vegetables that costs \$0.65 per pound?
\end{frame}

\begin{frame}
  \frametitle{Mixture and Percentage Word Problems}
  {\ubung} Two hundred litres of a punch that contains 35\% fruit
  juice is mixed with 300 litres (L) of another punch. The resulting
  fruit punch is 20\% fruit juice. Find the percent of fruit juice in
  the 300 litres of punch.
\end{frame}

\begin{frame}
  \frametitle{Mixture and Percentage Word Problems}
  {\ubung} Ten grams of sugar are added to a 40-g serving of a
  breakfast cereal that is 30\% sugar. What is the percent
  concentration of sugar in the resulting mixture?
\end{frame}

\begin{frame}
  \frametitle{Mixture and Percentage Word Problems}
  {\ubung} Your school is holding an event this weekend. Students have
  been pre-selling tickets to the event; adult tickets are \$5.00, and
  child tickets (for kids six years old and under) are \$2.50. From
  past experience, you expect about 13,000 people to attend the event.

\medskip

  This is the first year in which tickets prices have been reduced for
  the younger children, so you really don't know how many child
  tickets and how many adult tickets you can expect to sell. You
  decide to use the information from the pre-sold tickets to estimate
  the ratio of adults to children, and estimate the expected revenue
  from this information.
\end{frame}

\begin{frame}
  \frametitle{Mixture and Percentage Word Problems}
  You consult with your student ticket-sellers and discover that they
  have not been keeping track of how many child tickets they have
  sold. The tickets are identical, until the ticket-seller punches a
  hole in the ticket, indicating that it is a child ticket. They don't
  remember how many holes they have punched. They only know that they
  have sold 548 tickets for \$2460. How much revenue from each of
  child and adult tickets can you expect?
\end{frame}

\begin{frame}
  \frametitle{Mixture and Percentage Word Problems}
{\ubung} Two different mixtures of gasohol are available,
one with 5\% alcohol and the other containing 12\% alcohol. How many
gallons of the 12\% mixture must be added to 252 gallons of the 5\%
mixture to produce a mixture containing 9\% alcohol?
\end{frame}

\begin{frame}
  \frametitle{Mixture and Percentage Word Problems}
{\ubung} 15 litres of fuel containing 3.2\% oil is
  available for a certain two-cycle engine. This fuel is to be used
  for another engine requiring a 5.5\% oil mixture. How many litres of
  oil must be added?
\end{frame}

\begin{frame}
  \frametitle{Mixture and Percentage Word Problems}
{\ubung} How many litres of a solution containing 18\%
  sulfuric acid and how many litres of another solution containing
  25\% sulfuric acid must be mixed together to make 552 litres of
  solution containing 23\% sulfuric acid? (All percentages are by
  volume.)
\end{frame}

\begin{frame}
  \frametitle{Mixture and Percentage Word Problems}
  {\ubung} How many kilograms of brass containing 63\% copper must be
  melted with 1120kg of brass containing 72\% copper to produce a new
  brass containing 67\% copper?
\end{frame}

\begin{frame}
  \frametitle{End of Presentation}
Next Lesson: Exponential Functions
\end{frame}

\end{document}

