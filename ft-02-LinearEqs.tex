% ft-02-LinearEqs.tex

\documentclass[xcolor=dvipsnames]{beamer}

\usepackage{cancel}
\renewcommand{\CancelColor}{\color{red}}
\usepackage{graphicx}
\usepackage{wrapfig}
\usepackage{colortbl}
\usepackage{color}
\usepackage{alltt}
\renewcommand*{\thefootnote}{\fnsymbol{footnote}}
\definecolor{myblue}{rgb}{0.8,0.85,1}

\mode<presentation>
{
  \usetheme{Warsaw}
  \setbeamercovered{transparent}
}
% \usecolortheme[named=OliveGreen]{structure}
\setbeamertemplate{navigation symbols}{} 
\setbeamertemplate{blocks}[rounded][shadow=true] 

% this is for overlaying math symbols, see https://tex.stackexchange.com/questions/12895/overlay-symbol-with-another
\def\qeq{\mathrel{%
    \mathchoice{\QEQ}{\QEQ}{\scriptsize\QEQ}{\tiny\QEQ}%
}}
\def\QEQ{{%
    \setbox0\hbox{$\longrightarrow$}%
    \rlap{\hbox to \wd0{\hss/\hss}}\box0
  }}

\newcounter{expls}
\setcounter{expls}{0}
\newcommand{\beispiel}[1]{\refstepcounter{expls}\textbf{Example \arabic{expls}: #1.}}

\newcounter{exercise}
\setcounter{exercise}{0}
\newcommand{\ubung}[0]{\refstepcounter{exercise}\textbf{Exercise \arabic{exercise}: }}

\newif\ifBCITCourse
\BCITCoursetrue
% \BCITCoursefalse
\newif\ifWhichCourse
\WhichCoursetrue
% \WhichCoursefalse
\ifBCITCourse
\ifWhichCourse
\newcommand{\CourseName}{Technical Mathematics for Food Technology}
\newcommand{\CourseNumber}{MATH 1441}
\newcommand{\CourseInst}{BCIT}
\else
\newcommand{\CourseName}{Technical Mathematics for Geomatics}
\newcommand{\CourseNumber}{MATH 1511}
\newcommand{\CourseInst}{BCIT}
\fi
\else
\newcommand{\CourseName}{Philosophy and Literature}
\newcommand{\CourseNumber}{PHIL 375}
\newcommand{\CourseInst}{UBC}
\fi

\title{Set Theory, Functions, Numbers}
\subtitle{{\CourseNumber}, BCIT}

\author{\CourseName}

\date{September 7, 2017}

\begin{document}

\begin{frame}
  \titlepage
\end{frame}

\begin{frame}
  \frametitle{Equations}
  {\ubung} Determine the solution set.
\begin{equation}
  \label{eq:b2}
  \begin{array}{rcl}
    8+x&=&13 \\ 
    x^{2}&=&4 \\ 
    \frac{x}{1}&=&x \\ 
    x+2&=&x \\ 
    \frac{x-7}{x-7}&=&1 \\ 
    \sqrt{x+1}&=&x-5 \\ 
  \end{array}\notag
\end{equation}
\end{frame}

\begin{frame}
  \frametitle{Equations}
  \addtocounter{exercise}{-1}
  {\ubung} Determine the solution set.
\begin{equation}
  \label{eq:b2a}
  \begin{array}{rcll}
    8+x&=&13&\alert{S=\{5\}} \\ 
    x^{2}&=&4&\alert{S=\{-2,2\}} \\ 
    \frac{x}{1}&=&x&\alert{S=\mathbb{R}} \\ 
    x+2&=&x&\alert{S=\{\}} \\ 
    \frac{x-7}{x-7}&=&1&\alert{S=\mathbb{R}\setminus\{7\}} \\ 
    \sqrt{x+1}&=&x-5&\alert{S=\{8\},\mbox{ NOT }S=\{3,8\}} \\ 
  \end{array}\notag
\end{equation}
\end{frame}

\begin{frame}
  \frametitle{Linear Equations}
An equation is said to be linear if the variable appears at most to the
power of $1$. Here are some examples,
\begin{equation}
  \label{eq:oothahlo}
  \begin{array}{rcl}
8x-6&=&12 \\
\hspace{.5in} && \\
3(p-5)&=&8 \\
\hspace{.5in} && \\
4-3(t-5)&=&9t
  \end{array}
\end{equation}
\end{frame}

\begin{frame}
  \frametitle{Linear Equations}
An equation is said to be linear if the variable appears at most to the
power of $1$. Here are some examples,
\begin{equation}
  \label{eq:auxaiboz}
  \begin{array}{rcll}
8x-6&=&12&\alert{S=\left\{\frac{9}{4}\right\}} \\
\hspace{.5in} &&& \\
3(p-5)&=&8&\alert{S=\left\{\frac{23}{3}\right\}} \\
\hspace{.5in} &&& \\
4-3(t-5)&=&9t&\alert{S=\left\{\frac{19}{12}\right\}}
  \end{array}
\end{equation}
\end{frame}

\begin{frame}
  \frametitle{Doing the Same Thing to Both Sides I}
Here is a proof that $1=2$. Let $a$ and $b$ be some real numbers for
which we know that they are not zero and that they are equal, so
$a,b\neq{}0$ and $a=b$. Then
\begin{equation}
  \label{eq:fdlsjfjj}
  \begin{array}{rclcl}
    a&=&b&|&\cdot{}a \\
    a^{2}&=&ab&|&-b^{2} \\
    a^{2}-b^{2}&=&ab-b^{2}&|&\mbox{factor} \\
    (a+b)(a-b)&=&b(a-b)&|&\div{}(a-b) \\
    a+b&=&b&|&\mbox{replace }a\mbox{ by }b \\
    b+b&=&b&|&\mbox{simplify} \\
    % 2b&=&b&|&\div{}b\mbox{ note that }b\neq{}0 \\
    2b&=&b&|&\div{}b \\
    2&=&1&& \\
  \end{array}
\end{equation}
\end{frame}

\begin{frame}
  \frametitle{Doing the Same Thing to Both Sides II}
  The key to solving equations is to \alert{do the same thing to both
    sides}. Let $A,B,D$ be any mathematical expressions. Then
\begin{equation}
  \label{eq:iughaijo}
  A=B
\end{equation}
is equivalent to
\begin{equation}
  \label{eq:ieraechi}
  \begin{array}{rcl}
    A+D&=&B+D \\
    A-D&=&B-D \\
    A\cdot{}D&=&B\cdot{}D \\
    \frac{A}{D}&=&\frac{B}{D} \\
  \end{array}
\end{equation}
although for the latter two it is important that $D\neq{}0$,
otherwise the relevant function $F$ applied to both sides is not
bijective.
\end{frame}

\begin{frame}
  \frametitle{Doing the Same Thing to Both Sides III}
Are the following also equivalent to $A=B$?
\begin{equation}
  \label{eq:uozingei}
  \begin{array}{rcl}
    A^{2}&=&B^{2} \\
\hspace{.5in} && \\
    |A|&=&|B| \\
\hspace{.5in} && \\
    \sqrt{A}&=&\sqrt{B}
  \end{array}
\end{equation}
\end{frame}

\begin{frame}
  \frametitle{Doing the Same Thing to Both Sides III}
Are the following also equivalent to $A=B$?
\begin{equation}
  \label{eq:upohngee}
  \begin{array}{rcll}
    A^{2}&=&B^{2}&\alert{\mbox{no, use with caution}} \\
\hspace{.5in} &&& \\
    |A|&=&|B|&\alert{\mbox{no, use with caution}} \\
\hspace{.5in} &&& \\
    \sqrt{A}&=&\sqrt{B}&\alert{\mbox{no, use with caution}}
  \end{array}
\end{equation}
\end{frame}

\begin{frame}
  \frametitle{Doing the Same Thing to Both Sides IV}
Consider the following:
\begin{equation}
  \label{eq:oopiesho}
  \begin{array}{rcl}
    (x-1)^{2}&=&4 \\
\hspace{.5in} && \\
    |x-1|&=&4 \\
\hspace{.5in} && \\
    \sqrt{21-4x}&=&x
  \end{array}
\end{equation}
\end{frame}

\begin{frame}
  \frametitle{Doing the Same Thing to Both Sides IV}
Consider the following:
\begin{equation}
  \label{eq:peihahto}
  \begin{array}{rcll}
    (x-1)^{2}&=&4&\alert{S=\{-1,3\}} \\
\hspace{.5in} &&& \\
    |x-1|&=&4&\alert{S=\{-3,5\}} \\
\hspace{.5in} &&& \\
    \sqrt{21-4x}&=&x&\alert{S=\{3\}}
  \end{array}
\end{equation}
For the last equation, $S=\{3\}$ even though the corresponding
quadratic equation $x^{2}+4x-21=0$ has as its solutions $\{-7,3\}$.
\end{frame}

\begin{frame}
  \frametitle{Linear Equations with Fractions}
When the equation contains fractions, it is helpful to remember prime
number factorization and the greatest common denominator.
\begin{equation}
  \label{eq:sheekahf}
  \begin{array}{rcl}
\frac{p}{4}&=&\frac{7}{8}+\frac{2p}{3} \\
\hspace{.5in} && \\
\frac{6y}{7}&=&\frac{4}{9}y-\frac{1}{4} \\
  \end{array}
\end{equation}
\end{frame}

\begin{frame}
  \frametitle{Linear Equations with Fractions}
When the equation contains fractions, it is helpful to remember prime
number factorization and the greatest common denominator.
\begin{equation}
  \label{eq:xiupaete}
  \begin{array}{rcll}
\frac{p}{4}&=&\frac{7}{8}+\frac{2p}{3}&\alert{S=\left\{-\frac{21}{10}\right\}} \\
\hspace{.5in} && \\
\frac{6y}{7}&=&\frac{4}{9}y-\frac{1}{4}&\alert{S=\left\{-\frac{63}{104}\right\}} \\
  \end{array}
\end{equation}
\end{frame}

\begin{frame}
  \frametitle{Cross-Multiplying I}
Another excellent way to get rid of fractions is to cross-multiply.
Cross-multiplying means that if $B,D\neq{}0$ then the equation
\begin{equation}
  \label{eq:queighaw}
  \frac{A}{B}=\frac{C}{D}
\end{equation}
is equivalent to the equation
\begin{equation}
  \label{eq:maipahlu}
  A\cdot{}D=B\cdot{}C
\end{equation}
\end{frame}

\begin{frame}
  \frametitle{Cross-Multiplying II}
Here is an example.
\begin{equation}
  \label{eq:oucaedoo}
  \begin{array}{rclcl}
    \frac{x+1}{x-7}&=&-\frac{3}{5}&|&\mbox{cross-multiply} \\
    5(x+1)&=&(-3)(x-7)&|&\mbox{expand} \\
    5x+5&=&-3x+21&|&+3x-5 \\
    8x&=&16&|&\div{}8 \\
    x&=&2&&
  \end{array}
\end{equation}
Therefore, $S=\{2\}$.
\end{frame}

\begin{frame}
  \frametitle{Exercises Linear Equations}
{\ubung} Solve the following equations,
\begin{equation}
  \label{eq:b1}
  \begin{array}{rcl}
    -7w&=&15-2w \\ 
    \hspace{.5in} && \\
    \frac{z}{5}&=&\frac{3}{10}z+7 \\ 
    \hspace{.5in} && \\
    4\left(y-\frac{1}{2}\right)-y&=&6(5-y) \\ 
    \hspace{.5in} && \\
    5(x+3)+9&=&-2(x-2)-1
  \end{array}
\end{equation}
\end{frame}

\begin{frame}
  \frametitle{Exercises Linear Equations}
{\ubung} Solve the following equations,
\begin{equation}
  \label{eq:b1}
  \begin{array}{rcl}
    5t-13&=&12-5t \\ 
    \hspace{.5in} && \\
    2(1-x)&=&3(1+2x)+5 \\ 
    \hspace{.5in} && \\
    \frac{1}{2}y-2&=&\frac{1}{3}y \\ 
    \hspace{.5in} && \\
    \frac{2}{3}y+\frac{1}{2}(y-3)&=&\frac{y+1}{4} \\ 
  \end{array}
\end{equation}
\end{frame}

\begin{frame}
  \frametitle{End of Lesson}
Next Lesson: Linear Equations
\end{frame}

\end{document}

