% ft-final-exam-instructions.tex

\documentclass[11pt]{article}
\usepackage{enumerate}
\usepackage{syllogism} 
\usepackage{october}
\usepackage[table]{xcolor}

\begin{document}

\textbf{Final Exam Instructions}

Here are the types of questions you can expect for the final exam.

\begin{itemize}
\item Solve any of the following equations
  \begin{itemize}
  \item linear
  \item quadratic
  \item exponential
  \item logarithmic
  \end{itemize}
\item Solve mixture problems.
\item Simplify algebraic expressions, especially radicals, exponents,
  and logarithms. Algebraic expressions are expressions containing
  variables such as $x,y$, etc.
\item Solve problems involving interest rates, exponential growth,
  exponential decay, or Newton's Cooling Law.
\item Differentiate functions and find line equations for tangent
  lines of a function at a point.
\item Analyze functions (required features will be listed in the
  question).
\item Use Newton's method to find the $x$-intercept(s) of a function.
\item Find definite and indefinite integrals and areas under a curve.
\item Solve word problems using any of the methods in this list.
\end{itemize}

Here are some exam instructions.

\begin{enumerate}
\item You are allowed to use any hand-held calculator you want. No
  devices that are not primarily used as calculators are permitted
  (such as a smartphone).
\item You must show your work. Results without work leading to those
  results do not earn points.
\end{enumerate}

Here are some problems to practice for the final exam.

\begin{enumerate}
\item Solve the following equations.
  \begin{equation}
    \label{eq:thephohk}
    \frac{7}{2}x-\frac{1}{3}=\frac{3}{4}
  \end{equation}
  \begin{equation}
    \label{eq:uchoowai}
    % Stewart Redlin Watson page 43 exercise 41
    \frac{\frac{5}{x-1}-\frac{2}{x+1}}{\frac{x}{x-1}+\frac{1}{x+1}}
  \end{equation}
  \begin{equation}
    \label{eq:eishahji}
    6y^{2}-2\sqrt{3}y-1=0
  \end{equation}
  \begin{equation}
    \label{eq:vohtovuj}
    25^{3x-2}=625^{2x+7}
  \end{equation}
  \begin{equation}
    \label{eq:laishedu}
    \log_{8}(x+1)-\log_{8}x=\log_{8}4
  \end{equation}
\item Portland cement contains 21.9\% silicon dioxide. 500 litres of
  it are mixed with 300 litres of slag cement. The resulting mixture
  has a silicon dioxide content of 26.8125\%. What is the silicon
  dioxide content of slag cement?
\item Simplify. Use root signs where possible and avoid negative
  exponents in your answer.
  \begin{equation}
    \label{eq:einicohx}
    \left(\frac{x^{\frac{2}{3}}}{4y^{-2}}\right)^{-\frac{1}{2}}
  \end{equation}
\item The Van Gogh painting \emph{Irises} sold for \$84,000 in 1947
  and was sold again in 1987 for \$53,900,000. Assuming that the
  growth in value $V$ of the painting was exponential, (i) determine the
  doubling time for the value, (ii) estimate the value of the painting
  in 2007, and (iii) provide the year in which the painting will be worth
  one billion dollars.
\item You take a rock out of the fireplace at 1pm and measure its
  temperature at 305$^{\circ}$C. Five minutes later you measure again,
  and the temperature is $198^{\circ}$C. If room temperature is
  21.4$^{\circ}$, then when will the temperature of the rock be
  100$^{\circ}$?
\item Analyze the following function, i.e.\ find $x$-intercepts,
  critical points (indicate whether they are maxima, minima, or
  neither), inflection points, domain and range. Are there any
  asymptotes? 
  \begin{equation}
    \label{eq:hiexaize}
    f(x)=x^{3}-2x
  \end{equation}
\item Find the $x$-intercept for the function $f(x)=2x^{3}-5x^{2}+2x-5$
using Newton's method. Begin with $x_{1}=n$, where $n$ is a whole
number. Precision: about four significant digits.
\item Find the $x$-intercept for $f(x)=x^{2}-8$, using Newton's method.
Begin with $x_{1}=n$, where $n$ is a whole number. Compare the number
to $2\sqrt{2}$ and make sure that the at least four significant digits
match.
\item Find one of the $x$-intercepts for the function
$f(x)=e^{x}-\ln{}x-3$. Precision: about four significant digits.
\item Solve the following indefinite integrals.
% Calter p721
\begin{equation}
  \label{eq:uleinaig}
  \int\frac{5x}{\sqrt{3x^{2}-7}}dx \notag
\end{equation}
\begin{equation}
  \label{eq:teenaeta}
  \int(x^{2}+2)^{3}2xdx\notag
\end{equation}
\begin{equation}
  \label{eq:eexinuli}
  \int(x^{3}+3)^{6}x^{2}dx\notag
\end{equation}
\begin{equation}
  \label{eq:eipheico}
  \int(x^{3}+3x)^{3}(x^{2}+1)dx\notag
\end{equation}
\begin{equation}
  \label{eq:neebazek}
  \int{}x\sqrt{x^{2}+3}dx\notag
\end{equation}
\begin{equation}
  \label{eq:ohfohbee}
  \int\frac{x}{3-x^{2}}dx\notag
\end{equation}
\begin{equation}
  \label{eq:aawozoog}
  \int{}\frac{7-x^{2}}{x}dx\notag
\end{equation}
\item Evaluate the following definite integrals.
\begin{equation}
  \label{eq:aethaeng}
  \int_{-2}^{2}x^{2}(x+2)dx\notag
\end{equation}
\begin{equation}
  \label{eq:eangekuu}
  \int_{2}^{4}(x+3)^{2}dx\notag
\end{equation}
\begin{equation}
  \label{eq:teighaqu}
  \int_{0}^{1}\frac{x}{\sqrt{2-x^{2}}}dx\notag
\end{equation}
\begin{equation}
  \label{eq:pohkeiza}
  \int_{1}^{e}\frac{1}{x}dx\notag
\end{equation}
\begin{equation}
  \label{eq:ohgahjah}
  \int_{1}^{2}\frac{x^{4}+x^{3}+1}{x^{3}}dx\notag
\end{equation}
\end{enumerate}

\end{document}

