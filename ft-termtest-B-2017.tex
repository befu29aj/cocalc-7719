% ft-termtest-B.tex

\documentclass[11pt]{article}
\usepackage{alltt}
\usepackage{enumerate}
\usepackage{syllogism} 
\usepackage{october}
\usepackage[table]{xcolor}
\pagestyle{empty}

\newif\ifOneOrTwo
\OneOrTwotrue
% \OneOrTwofalse
\ifOneOrTwo
\newcommand{\eibe}{1}
\newcommand{\usha}{\frac{3+x}{2}-\frac{2x-7}{3}=3}
\newcommand{\ufoj}{2(\log_{5}x+2\log_{5}y-3\log_{5}z)}
\newcommand{\mair}{\ln\frac{x^{3}\sqrt{x-1}}{3x+4}}
\newcommand{\utit}{You have 20 gallons of a 45\% antifreeze solution. How many gallons of a 57\% antifreeze solution needs to be added to make a 51\% antifreeze solution?}
\newcommand{\jief}{Suppose a car can run on ethanol and gas and you have a 15 gallons tank to fill. You can buy fuel that is either 30 percent ethanol or 80 percent ethanol. How much of each type of fuel should you mix so that the mixture is 40 percent ethanol?}
\newcommand{\caib}{}
\newcommand{\eizi}{3}
\newcommand{\afie}{8}
\newcommand{\uphe}{}
\newcommand{\kieg}{}
\newcommand{\aequ}{}
\newcommand{\aiza}{}
\newcommand{\ajoo}{}
\newcommand{\icoo}{}
\newcommand{\phae}{}
\newcommand{\oaga}{}
\newcommand{\cait}{}
\newcommand{\viob}{}
\newcommand{\mare}{}
\newcommand{\ieje}{}
\newcommand{\opie}{}
\newcommand{\geik}{}
\newcommand{\ahna}{}
\newcommand{\xohv}{}
\newcommand{\dahj}{}
\newcommand{\eexe}{}
\newcommand{\etah}{}
\newcommand{\ieja}{}
\newcommand{\vail}{}
\newcommand{\sien}{}
\newcommand{\ieti}{}
\newcommand{\ooga}{}
\newcommand{\oopu}{}

\else
\newcommand{\eibe}{2}
\newcommand{\usha}{\frac{4+x}{2}-\frac{3x-2}{5}=2}
\newcommand{\ufoj}{\frac{1}{3}\log(2x+1)+\frac{1}{2}[\log(x-4)-\log(x^{4}-x^{2}-1)]}
\newcommand{\mair}{\ln\frac{10^{x}}{x(x^{2}+1)(x^{4}+2)}}
\newcommand{\utit}{You have 6 liters of water that have 20 percent strawberry juice. How many liters of a 80 percent strawberry juice should be added to the mixture to make 75 percent strawberry juice?}
\newcommand{\jief}{Solution X is a 27\% salt solution and Solution Y is a 20\% salt solution. How much of each is needed to make 42 gallons of a 25\% salt solution?}
\newcommand{\caib}{1}
\newcommand{\eizi}{2}
\newcommand{\afie}{3}
\newcommand{\uphe}{}
\newcommand{\kieg}{}
\newcommand{\aequ}{}
\newcommand{\aiza}{}
\newcommand{\ajoo}{}
\newcommand{\icoo}{}
\newcommand{\phae}{}
\newcommand{\oaga}{}
\newcommand{\cait}{}
\newcommand{\viob}{}
\newcommand{\mare}{}
\newcommand{\ieje}{}
\newcommand{\opie}{}
\newcommand{\geik}{}
\newcommand{\ahna}{}
\newcommand{\xohv}{}
\newcommand{\dahj}{}
\newcommand{\eexe}{}
\newcommand{\etah}{}
\newcommand{\ieja}{}
\newcommand{\vail}{}
\newcommand{\sien}{}
\newcommand{\ieti}{}
\newcommand{\ooga}{}
\newcommand{\oopu}{}
\fi

\newcounter{aufg}
\setcounter{aufg}{0}
\newcommand{\aufgabe}[1]{\refstepcounter{aufg}\textbf{(\arabic{aufg})}
[#1 points]}

\begin{document}

\textbf{Term Test B}

\aufgabe{6} Solve the following three equations. Remember that the
logarithm of a non-positive number is not defined.
\begin{equation}
  \label{eq:quaenaes}
  3^{2x-1}=27\notag
\end{equation}
\begin{equation}
  \label{eq:iegeevei}
  \log_{2}x+\log_{2}(x-2)=3\notag
\end{equation}
\begin{equation}
  \label{eq:uphubuze}
  \ln(x+1)^{2}=2\notag
\end{equation}

\aufgabe{5} How long will it take the world population to double at an
exponential growth rate of 1.64\% per year?

\aufgabe{5} Suppose we are preparing a lovely \emph{Canard {\`a}
  l'Orange} (roast duck with orange sauce). We first take our duck out
of a 36$^{\circ}$F refrigerator and place it in a 350$^{\circ}$F oven
to roast. After 10 minutes the internal temperature is 53$^{\circ}$F.
If we want to roast the duck until just under well-done (about
170$^{\circ}$F internally), when will it be ready?

\aufgabe{3} Evaluate without a calculator. Show all of your work.
\begin{equation}
  \label{eq:aushoote}
  \log_{4}\left(2\cdot\sqrt{32}\right)+\log_{27}\sqrt{3}\notag
\end{equation}

\aufgabe{6} Solve the following three equations.
\begin{equation}
  \label{eq:eiciecah}
  7^{x-5}=2\notag
\end{equation}
\begin{equation}
  \label{eq:vachaegi}
  \ln(5-2x)=-2\notag
\end{equation}
\begin{equation}
  \label{eq:ebeerohg}
  6-5e^{x}=-e^{2x}\notag
\end{equation}

\aufgabe{5} Suppose that you plan to need \$10,000 in thirty-six
months' time when your child starts attending university. You want to
invest in an instrument yielding 3.5\% interest, compounded monthly.
How much should you invest? Use the formula
\begin{equation}
  \label{eq:paeniawa}
  A=P\left(1+\frac{r}{m}\right)^{mt}
\end{equation}

\end{document}

