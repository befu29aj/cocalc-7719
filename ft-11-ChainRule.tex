% ft-11-ChainRule.tex

\documentclass[xcolor=dvipsnames]{beamer}

\usepackage{cancel}
\renewcommand{\CancelColor}{\color{red}}
\usepackage{graphicx}
\usepackage{wrapfig}
\usepackage{colortbl}
\usepackage{color}
\usepackage{alltt}
\renewcommand*{\thefootnote}{\fnsymbol{footnote}}
\definecolor{myblue}{rgb}{0.8,0.85,1}

\mode<presentation>
{
  \usetheme{Warsaw}
  \setbeamercovered{transparent}
}
% \usecolortheme[named=OliveGreen]{structure}
\setbeamertemplate{navigation symbols}{} 
\setbeamertemplate{blocks}[rounded][shadow=true] 

% this is for overlaying math symbols, see https://tex.stackexchange.com/questions/12895/overlay-symbol-with-another
\def\qeq{\mathrel{%
    \mathchoice{\QEQ}{\QEQ}{\scriptsize\QEQ}{\tiny\QEQ}%
}}
\def\QEQ{{%
    \setbox0\hbox{$\longrightarrow$}%
    \rlap{\hbox to \wd0{\hss/\hss}}\box0
  }}

\newcounter{expls}
\setcounter{expls}{0}
\newcommand{\beispiel}[1]{\refstepcounter{expls}\textbf{Example \arabic{expls}: #1.}}

\newcounter{exercise}
\setcounter{exercise}{0}
\newcommand{\ubung}[0]{\refstepcounter{exercise}\textbf{Exercise \arabic{exercise}: }}

\newif\ifBCITCourse
\BCITCoursetrue
% \BCITCoursefalse
\newif\ifWhichCourse
\WhichCoursetrue
% \WhichCoursefalse
\ifBCITCourse
\ifWhichCourse
\newcommand{\CourseName}{Technical Mathematics for Food Technology}
\newcommand{\CourseNumber}{MATH 1441}
\newcommand{\CourseInst}{BCIT}
\else
\newcommand{\CourseName}{Technical Mathematics for Geomatics}
\newcommand{\CourseNumber}{MATH 1511}
\newcommand{\CourseInst}{BCIT}
\fi
\else
\newcommand{\CourseName}{Philosophy and Literature}
\newcommand{\CourseNumber}{PHIL 375}
\newcommand{\CourseInst}{UBC}
\fi

\title{Chain Rule}
\subtitle{{\CourseNumber}, BCIT}

\author{\CourseName}

\date{November 9, 2017}

\begin{document}

\begin{frame}
  \titlepage
\end{frame}

\begin{frame}
  \frametitle{Problematic Functions}
Here are some functions that we either don't know how to differentiate
or whose differentiation would take an inordinate amount of time.
\begin{equation}
  \label{eq:faegeehi}
f(x)=2^{x}
\end{equation}
\begin{equation}
  \label{eq:kooteiju}
f(x)=\sqrt{x^{2}+1}
\end{equation}
\begin{equation}
  \label{eq:oochahph}
f(x)=(x^{2}+x+1)^{100}
\end{equation}
\begin{equation}
  \label{eq:bongaeza}
f(x)=\sin(1+\sqrt{x-7})
\end{equation}
\begin{equation}
  \label{eq:ooquonge}
f(x)=\log_{10}x
\end{equation}
\begin{equation}
  \label{eq:iejafaic}
f(x)=\ln(x^{2}+1)
\end{equation}
\end{frame}

\begin{frame}
  \frametitle{The Chain Rule}
  \begin{block}{Rule 7}
The Chain Rule
  \end{block}
\begin{equation}
  \label{eq:aepuaxai}
g'(x)=f_{1}'(f_{2}(x))f_{2}'(x)\mbox{ for }g(x)=(f_{1}\circ{}f_{2})(x)
\end{equation}
\end{frame}

\begin{frame}
  \frametitle{Chain Rule Reason}
Consider
\begin{align}
  \label{eq:prf}
  (f\circ{}g)'(x)=\lim_{h\rightarrow{}0}\frac{f(g(x+h))-f(g(x))}{h}&=&\notag \\
  \lim_{h\rightarrow{}0}\frac{f(g(x+h))-f(g(x))}{g(x+h)-g(x)}\cdot\lim_{h\rightarrow{}0}\frac{g(x+h)-g(x)}{h}&=& \\
  f'(g(x))g'(x)&&
\end{align}
This is only a hint, not a rigorous proof, since we have replaced
$g(x+h)$ by $g(x)+h$, which isn't covered by our rules and is, in
fact, false in some situations.
\end{frame}

\begin{frame}
  \frametitle{Exercises}
  \begin{enumerate}
  \item<1-> Diffentiate: $f(x)=2^{x}$
  \item<2-> Diffentiate: $f(x)=\sqrt{x^{2}+1}$
  \item<3-> Diffentiate: $f(x)=(x^{2}+x+1)^{100}$
  \item<4-> Diffentiate: $f(x)=\log_{10}x$
  \item<5-> Diffentiate: $f(x)=\ln(x^{2}+1)$
  \end{enumerate}
\end{frame}

\begin{frame}
  \frametitle{Inverse and Identity Function}
Remember how we defined the logarithmic function,
\begin{equation}
  \label{eq:tieteiph}
  \ln{}y=x\mbox{ if and only if }e^{x}=y
\end{equation}
so the logarithmic function is the inverse of the exponential
function. Consequently, if $f(x)=e^{x}$ and $g(y)=\ln{}y$
\begin{equation}
  \label{eq:iewaebef}
  (f\circ{}g)(y)=y\mbox{ and }(g\circ{}f)(x)=x
\end{equation}
When (\ref{eq:iewaebef}) is true we call $f$ the \alert{inverse
  function} of $g$ and vice versa. The function $\mbox{id}(x)=x$ is called
the \alert{identity function}. 
\end{frame}

\begin{frame}
  \frametitle{The Derivative of the Exponential Function}
We know the derivative of the identity function.
\begin{equation}
  \label{eq:sivahzuw}
  \mbox{id}'(x)=1
\end{equation}
Consequently,
\begin{equation}
  \label{eq:zeejaixu}
  \frac{d}{dx}\ln\left(e^{x}\right)=1
\end{equation}
We also know that according to the chain rule
\begin{equation}
  \label{eq:iecheixo}
  \frac{d}{dx}\ln\left(e^{x}\right)=\frac{1}{e^{x}}\exp'(x)
\end{equation}
where $\exp(x)=e^{x}$. Therefore,
\begin{equation}
  \label{eq:zigaewai}
  \exp'(x)=e^{x}
\end{equation}
The exponential function is its own derivative!
\end{frame}

\begin{frame}
  \frametitle{Derivative of the Exponential Function: Exercises}
Differentiate the following functions:
\begin{equation}
  \label{eq:beetulae}
f(x)=e^{\sin{}x}  
\end{equation}
\begin{equation}
  \label{eq:aekephii}
g(t)=\frac{1}{e^{t}}  
\end{equation}

\bigskip

\begin{equation}
  \label{eq:uijeabai}
v(w)=w^{2}e^{w}  
\end{equation}

\bigskip

\begin{equation}
  \label{eq:ohzabeed}
g(z)=\frac{e^{z}-1}{e^{z}+1}  
\end{equation}
\end{frame}

\begin{frame}
  \frametitle{Exercises for Differentiation I}
Differentiate the following functions or find $dy/dx$ for the
following curves:
\begin{equation}
  \label{eq:pimexeiz}
  f(\vartheta)=\tan(\sin{}\vartheta)
\end{equation}
\begin{equation}
  \label{eq:oogheica}
  F(x)=\sqrt[4]{1+2x+x^{3}}
\end{equation}
\begin{equation}
  \label{eq:ibagheab}
  g(t)=\frac{\pi}{(t^{4}+1)^{3}}
\end{equation}
\begin{equation}
  \label{eq:oboohoca}
  f(s)=\sqrt[3]{1+\tan{}s}
\end{equation}
\begin{equation}
  \label{eq:choopaib}
  y=(x^{2}+1)\sqrt[3]{x^{2}+2}
\end{equation}
\begin{equation}
  \label{eq:uosiamei}
y=e^{x\cos{}x}  
\end{equation}
\begin{equation}
  \label{eq:oshaiphu}
  y=x\sin\frac{1}{x}
\end{equation}
\end{frame}

\begin{frame}
  \frametitle{Exercises for Differentiation II}
Differentiate the following functions or find $dy/dx$ for the
following curves:
\begin{equation}
  \label{eq:ciukaech}
  y=3\cot(nx)
\end{equation}
\begin{equation}
  \label{eq:oveagooy}
  y=xe^{-kx}
\end{equation}
\begin{equation}
  \label{eq:veeveema}
  h(t)=(t^{4}-1)^{3}(t^{3}+1)^{4}
\end{equation}
\begin{equation}
  \label{eq:athaazui}
  y=(x^{2}+1)\sqrt{x^{2}+2}
\end{equation}
\begin{equation}
  \label{eq:ahgoovim}
  G(y)=\left(\frac{y^{2}}{y+1}\right)^{5}
\end{equation}
\begin{equation}
  \label{eq:gaidaime}
  y=\tan^{2}(3\vartheta)
\end{equation}
\end{frame}

\begin{frame}
  \frametitle{Exercises for Differentiation III}
Find an equation of the tangent line to the curve
\begin{equation}
  \label{eq:vaixohga}
  y=\frac{2}{1+e^{-x}}
\end{equation}
at $x=0$.

\bigskip

Here is a model for the length of daylight (in hours) in Toronto on
the $t$-th day of the year
\begin{equation}
  \label{eq:iefeuvae}
  L(t)=12+2.8\sin\left(\frac{2\pi}{365}(t-80)\right)
\end{equation}
Compare how the number of hours of daylight is increasing in Toronto
on March 21 and May 21.
\end{frame}

\begin{frame}
  \frametitle{End of Lesson}
Next Lesson: Optimization
\end{frame}

\end{document}

