% ft-lab-01.tex

\documentclass[11pt]{article}
\usepackage{enumerate}
\usepackage{syllogism} 
\usepackage{october}
\usepackage[table]{xcolor}

\newcounter{aufg}
\setcounter{aufg}{0}
\newcommand{\aufgabe}[0]{\refstepcounter{aufg}\textbf{(\arabic{aufg})}}

\begin{document}

\textbf{Linear Equations}

% Josef Laub: Lehrbuch der Mathematik, 1. Band, page 113
{\aufgabe} Solve the following equation:
\begin{equation}
  \label{eq:iexaigee}
  3x+7-(5-2x)=6x-2
  % 4
\end{equation}

{\aufgabe} Solve the following equation:
\begin{equation}
  \label{eq:iexaigee}
  7-3(2t-5)+5(9-3t)=5(1-7t)-2(4-3t)
  % -35/4
\end{equation}

{\aufgabe} Solve the following equation:
\begin{equation}
  \label{eq:iexaigee}
  (3x-5)^{2}-(2x+3)^{2}=5(x-2)(x+2)-2(14x+3)
  % 3
\end{equation}

{\aufgabe} Solve the following equation:
\begin{equation}
  \label{eq:iexaigee}
  (3y-2)(y-4)-(2y-3)^{2}+y(y-3)=7(y-1)-6(y+2)
  % 3
\end{equation}

{\aufgabe} Solve the following equation:
\begin{equation}
  \label{eq:iexaigee}
  99-10t(1-2t)=8t(t-1)^{2}-(2t-3)^{3}
  % -2
\end{equation}

{\aufgabe} Solve the following equation:
\begin{equation}
  \label{eq:iexaigee}
  \frac{3+5x}{8}-\frac{2(3x-1)}{4}=\frac{1+x}{2}+10
  % -7
\end{equation}

{\aufgabe} Solve the following equation:
\begin{equation}
  \label{eq:iexaigee}
  x+\frac{x+3}{2}-\frac{x}{4}-1=\frac{x+4}{8}
  % 0
\end{equation}

{\aufgabe} Solve the following equation:
\begin{equation}
  \label{eq:iexaigee}
  \frac{s}{2}+\frac{s}{3}-\frac{s}{4}=\frac{7s+5}{12}
  % {}
\end{equation}

{\aufgabe} Solve the following equation:
\begin{equation}
  \label{eq:iexaigee}
\frac{x}{x-1}+\frac{x+1}{x}=\frac{2x^{2}+3x-3}{x^{2}-x}
\end{equation}

{\aufgabe} Solve the following equation:
\begin{equation}
  \label{eq:iexaigee}
\frac{1}{x(x-1)}-\frac{2}{(x+1)(x-1)}=\frac{1}{x(x+1)}
\end{equation}

\end{document}

